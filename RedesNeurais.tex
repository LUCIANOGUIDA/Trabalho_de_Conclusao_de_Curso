\documentclass[a4paper,12pt]{article}

\usepackage[utf8]{inputenc}
\usepackage[portuguese]{babel}
\usepackage[table]{xcolor}
\usepackage{fancyhdr}
\usepackage{indentfirst}
\usepackage[alf]{abntex2cite}


% padrao 1.5 de espacamento entre linhas


\pagestyle{fancy}


\title{Projeto de Trabalho de Conclusão de Curso}
\author{Luciano Guida Damasceno}

\setstretch{1.5}
\begin{document}

\maketitle


\newpage


%################
%TEMA DO PROJETO
%################

\section{Tema}
Uso de redes neurais artificiais para previsão de séries temporais
\subsection{Título}
Desenvolvimento de uma rede neural artificial para previsão do preço da soja


\newpage

%################
%PROBLEMA DO PROJETO
%################

\section{Problema}
A proposta do seguinte projeto é desenvolver uma solução que facilite a descoberta de preços futuros através de redes neurais artificiais. A complexidade e riscos de prever mercados futuros sem auxilio de um método ou ferramenta adequada serão muitos maiores. Com isso uma previsão incorreta poderá acarretar em grandes prejuízos aos negociadores financeiros.

\newpage

%################
%JUSTIFICATIVA DO PROJETO
%################

\section{Justificativa}
\indent O uso de redes neurais é uma técnica muito poderosa, pois podem trabalhar com informações incompletas, aprender a partir de experiências, reconhecimento de padrões e previsão. A previsão de séries temporais é um método bastante promissor, principalmente por se tratar de analises históricos e de previsões. Aplicada à área como o agronegócio é de grande utilidade principalmente para auxiliar na tomada de decisões, onde investidores terão estimativas futuras, e com isso saberão se é viável ou não investir no período atual ou futuramente. 




\newpage


%################
%OBEJETIVOS DO PROJETO
%################

\section{Objetivos}

\subsection{Objetivo Geral}

O projeto tem como objetivo geral criar uma rede neural artificial capaz de analisar históricos do preço da soja e com isso ter como resultados previsões de preço.

\subsection{Objetivos Específicos}
\begin{itemize}
 \item Realizar pesquisa bibliográfica
 \item Obter dados históricos do preço da soja.
 \item Implementar rede neural;
 \item Comparar resultados com outros trabalhos.
\end{itemize}


\newpage

%################
%REVISÃO BIBLIOGRÁFICA DO PROJETO
%################

\section{Revisão bibliográfica}


\newpage


%################
%METODOLOGIA DO PROJETO
%################


\section{Metodologia}



\newpage


%################
%CRONOGRAMA DO PROJETO
%################

\section{Cronograma}

\begin{table}[h]

\centering
\setlength{\tabcolsep}{3pt} 

\footnotesize{
  \begin{tabular}{|p{3,6cm}|c|c|c|c|c|c|c|c|c|c|c|}

    \hline
    \small{Atividades} & FEV & MAR & ABR & MAI & JUN & AGO & SET & OUT & NOV & DEZ\\
    \hline
    
    Pesquisa Bibliográfica & \cellcolor{red} & \cellcolor{red} & \cellcolor{red} & \cellcolor{red} & \cellcolor{red} & & & & & \\
    \hline
    Implementação de um protótipo do algoritmo & & & & \cellcolor{red} & \cellcolor{red} & \cellcolor{red} & \cellcolor{red} & & & \\
    \hline
    Análise e teste de operadores do algoritmos implementado & & & & & & \cellcolor{red} & \cellcolor{red} & & & \\
    \hline
    Análise comparativa dos resultados entre o algoritmo implementado e o atual estado da arte & & & & & & \cellcolor{red} & \cellcolor{red} & \cellcolor{red} & & \\
    \hline
    Elaboração do Artigo Científico & & & \cellcolor{red} & \cellcolor{red} & \cellcolor{red} & \cellcolor{red} & \cellcolor{red} & \cellcolor{red} & & \\
    \hline
    Pre-defesa & & & & & & & & \cellcolor{red} & & \\
    \hline
    Apresentação e Publicação do Artigo & & & & & & & & & \cellcolor{red} & \\
    \hline
    Entrega final com as correções & & & & & & & & & & \cellcolor{red} \\  
    \hline

  \end{tabular}
}

\end{table}


\newpage

%################
%REFERÊNCIAS DO PROJETO
%################

\section{Referências Bibliográficas}
\bibliography{bibliografia}

\end{document}
